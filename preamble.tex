\usepackage[left=2.7cm,right=2.7cm,top=3.5cm,bottom=3cm]{geometry}
\usepackage{amsfonts}
\usepackage{amsthm}
\usepackage{amsmath}
\usepackage{parskip}
\usepackage{mathrsfs}
\usepackage{graphicx}
\usepackage{amssymb}
\usepackage{xtab}
\usepackage[english]{babel}	
\usepackage[latin1]{inputenc}
%\usepackage[notref,notcite]{showkeys}


% You need to set the next two items. 
% If you title is long, you may need to adjust the spacing in titlepage.tex
\newcommand{\TheAuthor}{Chen Wang}
\newcommand{\TheTitle}{Finite Dimensional C[x,y]-Module}

% uncomment exacly one of these
%\newcommand{\TheModule}{\bf MA4K8 Scholarly Report}
\newcommand{\TheModule}{\bf  Dissertation}


% Leave the following two as is
\newcommand{\TheUni}{The University of Warwick}
\newcommand{\TheDept}{Mathematics Institute}

% Set the correct submission year, e.g. replace 2000 with 2017
\newcommand{\TheSubDate}{\monthyear \formatdate{4}{9}{2021}}

% The are a lot of things you might find useful and want to uncomment. 
% Most things you will want to uncomment or change will be near the top.

% standard maths symbols (you probably want to uncomment all of these)
% \newcommand{\Z}{\ensuremath{\mathbb{Z}}}% integers
% \newcommand{\N}{\ensuremath{\mathbb{N}}}% natural numbers
% \newcommand{\R}{\ensuremath{\mathbb{R}}}% real numbers
% \newcommand{\C}{\ensuremath{\mathbb{C}}}% complex numbers

% Others 
%\newcommand{\st}{\ensuremath{:}}% such that
%\newcommand{\Tau}{\ensuremath{\mathcal{T}}}
%\newcommand{\Nat}{\mathbb{N}}

%\usepackage[numbers]{natbib}% round braces, sort multiple citations
%\usepackage{hyperref}% creates hypertext links in pdf files (natbib compatible)
\usepackage{setspace}
%\usepackage{fancyhdr}
%\usepackage{mathrsfs}
%\usepackage{textcomp}
%\usepackage{color}
%\usepackage{graphicx}
%\usepackage{framed}
% \usepackage{algorithmic}% format pseudocode
%\usepackage[vlined,boxed,commentsnumbered,algochapter]{algorithm2e}
% \usepackage[chapter]{algorithm}% float wrapper for algorithms
%\usepackage{amsmath}% American Mathematical Society macros - essential!
%\usepackage{amssymb}% contains amsfonts
%\usepackage{amsthm}% allows more flexibility with theorems
\usepackage[nodayofweek]{datetime} % change the format of printed dates (no american style!) 
%\usepackage{ifdraft}% perform operations conditional on the draft option

% \usepackage{mathptmx}
% \usepackage{mathpazo}
%\usepackage{amscd}
% \usepackage{xy}
% \usepackage{diagxy}
%\usepackage{diagrams}

%\usepackage{marginnote}
%\usepackage{rotating}
%\usepackage{multirow}
% \usepackage{polski}
% \usepackage[T1]{fontenc}
% \usepackage{tikzpicture}
% \usetikzlibrary{matrix,arrows}

% \usepackage[inline]{showlabels}
%\usepackage{booktabs}

% Date format

% Use the datetime package
\newdateformat{monthyear}{\monthname[\THEMONTH], \THEYEAR}% new date format

% New commands, operators and symbols

% Operators
%\DeclareMathOperator{\Sym}{Sym}% symmetric group
%\DeclareMathOperator{\Alt}{Alt}% alternating group
%\DeclareMathOperator{\Id}{Id}
%\DeclareMathOperator{\Hom}{Hom}
%\DeclareMathOperator{\Grp}{Grp}
\DeclareMathOperator{\supp}{supp}
\DeclareMathOperator{\pt}{pt}
%\DeclareMathOperator{\fix}{fix}
%\DeclareMathOperator{\dep}{dep}
%\DeclareMathOperator{\lcm}{lcm}
%\DeclareMathOperator{\Aut}{Aut}
%\DeclareMathOperator{\Inn}{Inn}
%\DeclareMathOperator{\Out}{Out}
% \DeclareMathOperator{\dim}{dim}
%\DeclareMathOperator{\Syl}{Syl}
%\DeclareMathOperator{\Hall}{Hall}
%\DeclareMathOperator{\pCore}{pCore}
%\DeclareMathOperator{\Char}{char}
%\DeclareMathOperator{\Image}{Im}
%\DeclareMathOperator{\Ker}{Ker}
%\DeclareMathOperator{\Hcf}{hcf}
%\DeclareMathOperator{\GL}{GL}
%\DeclareMathOperator{\Pc}{Pc}
%\DeclareMathOperator{\Stab}{Stab}
%\DeclareMathOperator{\Orbit}{Orbit}


% Hyphenation fixes
%\newcommand{\letdash}[1]{$#1$\nobreakdash-\hspace{0pt}}% for n-element, k-transitive etc
%\newcommand{\numdash}{\nobreakdash--}

% Sequences
%\newcommand{\seqfin}[3]{\ensuremath{#1_{#2}, \dotsc , #1_{#3}}}
%\newcommand{\seqinf}[3]{\ensuremath{#1_{#2}, #1_{#3}, \dotsc}}

% New environments

% Dedication
%\newenvironment{dedication}
%{\clearpage \thispagestyle{empty} \vspace*{\stretch{1}} \begin{center} \em}
%{\end{center} \vspace*{\stretch{3}} \clearpage}


% Page Layout

% Dimensions
% The guidlines for margins given by the Graduate School are as follows:
% left - no less than 4cm
% right - sufficient for trimming (?)
% top - no less than 1.5cm
% bottom - no less than 1.5cm
% See http://www2.warwick.ac.uk/services/academicoffice/gsp/current/presentation/
% Use the geometry package to set this up
\geometry{includehead,includefoot,left=3cm,right=3cm,top=2cm,bottom=2cm}% head and foot are included in total body so that nothing is printed out of the boundaries of the set margins

% Header and footer
% Use the fancydr package
%\ifdraft{%
%\fancypagestyle{plain}{% redefine plain pagestyle for draft
%\renewcommand{\headrulewidth}{0pt}% no head rule in draft mode
%\fancyhf{}% clear headers and footers
%\fancyhead[C]{\ifdraft{DRAFT}{}}% print DRAFT across top if the draft option is set
%\fancyfoot[C]{\thepage}% usual page numbering
%

%\pagestyle{fancy}
%\renewcommand{\chaptermark}[1]{\markboth{\thechapter.\ #1}{}}% compact with no capitalisation
%\renewcommand{\sectionmark}[1]{\markright{\thesection.\ #1}}% compact with no capitalisation
%\ifdraft{\renewcommand{\headrulewidth}{0pt}}{}% no head rule in draft mode
%\setlength{\headheight}{15pt}
%\fancyhf{}% clear all header and footer fields
% one-sided printing, so no E,O distinction need be made
%\fancyhead[C]{\ifdraft{DRAFT}{}}% print DRAFT across top if the draft option is set
%\fancyhead[R]{\thepage}% page in top right
%\fancyhead[L]{\ifdraft{}{\leftmark}}% chapter number and title in top left

% Theorems
%
%\newtheorem{prop}{Proposition}[chapter]
%\newtheorem{lemma}[prop]{Lemma}
%\newtheorem{conjecture}[prop]{Conjecture}
%\newtheorem{theorem}[prop]{Theorem}
%\newtheorem{hyp}[prop]{Hypothesis}
%\newtheorem{cor}[prop]{Corollary}
%\newtheorem{claim}[prop]{Claim}
%\newtheorem{remark}[prop]{Remark}
%\newtheorem{notation}[prop]{Notation}
%\newtheorem{defn}[prop]{Definition}


% Numbering
%\numberwithin{equation}{chapter}% standard style numbering, nothing special here


% Algorithms

% Use the algorithms package.
% \renewcommand{\algorithmicrequire}{\textbf{Input:}}
% \renewcommand{\algorithmicensure}{\textbf{Output:}}
% \renewcommand{\algorithmiccomment}[1]{/* #1 */}
%\RestyleAlgo{boxruled}
%\SetAlgoInsideSkip{medskip}
%\setlength{\algomargin}{2em}
%\setlength{\interspacetitleboxruled}{0.7em}
%\setlength{\interspacetitleruled}{0.7em}
%\LinesNumbered
%\SetAlCapNameSty{sc}
%\SetFuncSty{sc}

%\newenvironment{spacedalgorithm}{\begin{algorithm}\onehalfspacing}{\end{algorithm}}
% \newenvironment{onehalfverbatim}{\onehalfspacing\begin{verbatim}}{\end{verbatim}}

%\reversemarginpar
%\newcounter{nootje}
%\setcounter{nootje}{1}
%\renewcommand\check[1]{[*\thenootje]\marginnote{\tiny\begin{minipage}{40mm}\begin{flushright}\thenootje
%: #1\end{flushright}\end{minipage}}\addtocounter{nootje}{1}}


% Document details

%% Useful packages
\usepackage{amsmath}
\usepackage{graphicx}
\usepackage[colorinlistoftodos]{todonotes}
\usepackage[colorlinks,linkcolor=red,citecolor=black]{hyperref}
\hypersetup{ colorlinks=true, linkcolor=black }

\usepackage{bm}

\usepackage{amssymb} 
\usepackage{framed} 
\usepackage{pifont} 
\usepackage{enumerate} 
\usepackage{wasysym} 
\usepackage{amsthm}
\usepackage{comment}
\usepackage{graphicx}
\usepackage{amsfonts}
\usepackage{mathrsfs}
\usepackage{array}
\usepackage{url}
\usepackage{ulem}
\usepackage{multicol}
\usepackage{mathtools}
\usepackage{tikz-cd}
\usepackage{nicefrac}
\usepackage{etoolbox}
\patchcmd{\thebibliography}{\section*{\refname}}{}{}{}
\usepackage{cite}
\usepackage{xfrac}
\usepackage{faktor}
\usepackage{mathtools}
\usepackage{ulem}

\usepackage{stmaryrd}
\usepackage{amsthm}
\usepackage{lipsum} % for filler text

\newtheorem{theorem}{Theorem}

\newtheorem{thm}{Theorem}%[section] 
\newtheorem{lem}[thm]{Lemma}
\newtheorem{claim}{Claim}[thm]
\newtheorem{prop}[thm]{Proposition}
\newtheorem{cor}[thm]{Corollary}
\newtheorem{exe}[thm]{Example}
\newtheorem{conj}[thm]{Conjecture}

\theoremstyle{definition}
\newtheorem{defn}[thm]{Definition}%[section] 
\newtheorem{eg}{Example}%[section] 
\newtheorem{rem}{Remark} %[section] 

\theoremstyle{remark} 
\newtheorem*{note}{Note} 
\newtheorem{case}{Case}[thm]

% Commands for the document
\DeclareMathOperator{\Rep}{Rep}
\DeclareMathOperator{\GL}{GL}
\DeclareMathOperator{\SL}{SL}
\DeclareMathOperator{\Hom}{Hom}
\DeclareMathOperator{\End}{End}
\DeclareMathOperator{\Tor}{Tor}
\DeclareMathOperator{\Ext}{Ext}
\DeclareMathOperator{\Aut}{Aut}
\DeclareMathOperator{\Stab}{Stab}
\DeclareMathOperator{\rad}{rad}
\DeclareMathOperator{\fr}{frac}
\DeclareMathOperator{\coker}{coker}
\DeclareMathOperator{\im}{im}
\DeclareMathOperator{\id}{id}
\DeclareMathOperator{\pd}{pd}
\DeclareMathOperator{\glob}{gl.dim}


\renewcommand{\ge}{\geqslant}
\renewcommand{\le}{\leqslant}
\newcommand{\field}[1]{\mathbb{#1}}
\newcommand{\gal}{\mathrm{Gal}}
\newcommand{\Q}{\field{Q}}
\newcommand{\C}{\field{C}}
\newcommand{\R}{\field{R}}
\newcommand{\N}{\field{N}}
\newcommand{\Z}{\field{Z}}
\newcommand{\A}{\field{A}}
\newcommand{\Pp}{\field{P}}
\newcommand{\F}{\field{F}}
\newcommand{\K}{\field{K}}
\newcommand{\f}{\mathfrak}
\newcommand{\fp}{\mathfrak{p}}
\newcommand{\fq}{\mathfrak{q}}
\newcommand{\fa}{\mathfrak{a}}
\newcommand{\fb}{\mathfrak{b}}
\newcommand{\fm}{\mathfrak{m}}
\newcommand{\Ht}{\mathrm{height}}
\newcommand{\M}{M}

\newcommand{\bigquo}[2]{{\raisebox{.2em}{$#1$}\left/\raisebox{-.2em}{$#2$}\right.}}
\usepackage{faktor}
\usepackage[all]{xy}
\newcommand{\<}{\left<}
\DeclareMathOperator{\ass}{Ass}
\DeclareMathOperator{\maxs}{maxSpec}
\DeclareMathOperator{\spec}{Spec}
\DeclareMathOperator{\ann}{Ann}
\DeclareMathOperator{\diag}{diag}
\DeclareMathOperator{\hilb}{Hilb}
\DeclareMathOperator{\proj}{Proj}
\DeclareMathOperator{\ad}{ad}
\DeclareMathOperator{\Ad}{Ad}
\DeclareMathOperator{\Tr}{Tr}
\DeclareMathOperator{\Vect}{Vect}
\DeclareMathOperator{\d}{d}
\DeclareMathOperator{\im}{im}
\DeclareMathOperator{\Mat}{Mat}
\DeclareMathOperator{\Hom}{Hom}


%\usepackage[dvipsnames]{xcolor}
%Defining colour with different models.
\definecolor{mypink1}{rgb}{0.858, 0.188, 0.478}
\definecolor{mypink2}{RGB}{219, 48, 122}
\definecolor{mypink3}{cmyk}{0, 0.7808, 0.4429, 0.1412}
\definecolor{mygray}{gray}{0.6}


